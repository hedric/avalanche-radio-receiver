% Richard Hedlund
% Dokumentation
% Examensarbete i teknisk fysik

\documentclass[11pt, a4paper]{report}
\usepackage{array}
%\usepackage{fullpage}
\usepackage[left=1.5in, right=1in, top=1in, bottom=1in, includefoot, headheight=13.6pt]{geometry}
\usepackage{multirow}
\usepackage{amsmath}
\usepackage{amssymb}
\usepackage{graphicx}
\usepackage{epstopdf}
\usepackage{textcomp}
\usepackage[T1]{fontenc}
\usepackage{lmodern}
\usepackage[utf8]{inputenc}
\usepackage[swedish]{babel}
\usepackage{fixltx2e}
\usepackage{float}
\usepackage{listingsutf8}
\usepackage{fancyhdr}
\usepackage{parskip}
\usepackage{minted}
\usepackage{hyperref}
\newminted{python}{frame=lines,framerule=2pt}
\pagestyle{fancy}
\fancyhf{}
\lhead{Richard Hedlund \\ richard.hedlund@protonmail.com}
\rhead{073-0653632 \\ Dokumentation för radiomottagaren}
\begin{document}
\section*{Utvecklingsmiljö och mjukvara}
Operativsystemet  Linux har använts för hela utvecklingen. På min laptop har jag kört Linux-distributionen Ubuntu. På Odroid XU4 körs en lättare variant av Ubuntu som heter Mate. Terminalen öppnas genom start-menyn. Eller så kan man trycka på MOD (windows-knappen på tangentbordet) och sen skriva in namnet på det program man vill starta.

Det viktigaste som behövs installeras i en ny utvecklingsmiljö är följande
\begin{itemize}
    \item GNU Radio: \url{http://gqrx.dk/download/install-ubuntu}
    \item libairspyhf: \url{https://github.com/airspy/airspyhf}
    \item AirSpy HF+ support i GNU Radio: \url{https://github.com/rascustoms/gr-osmosdr}
    \item pymavlink: \url{https://mavlink.io/en}, där finns all information om hur det installeras och alla dependecies etc.
\end{itemize}
Länkarna ovan har instruktioner om hur man installerar.

\textbf{Generella kommandon och info:}

Root-lösenordet på Odroid är:
\begin{minted}[frame=single,framesep=10pt]{bash}
    odroid
\end{minted}
GNU Radios grafiska gränssnitt Gnu Radio Companion (GRC) har använts för signalbehandlingen. GRC startas genom att använda start-menyn eller genom terminalen och kommandot:
\begin{minted}[frame=single,framesep=10pt]{bash}
    gnuradio-companion
\end{minted}



För att avsluta körande processer i Linux, exempelvis de två körande programmen (DSP och detektion) kan göras enligt följande:
\begin{minted}[frame=single,framesep=10pt]{bash}
    killall -9 python2 dsp.py
    killall -9 python2 detection.py
\end{minted}

Eller helt enkelt:

\begin{minted}[frame=single,framesep=10pt]{bash}
    killall python2
\end{minted}

För att kunna göra ett Python-script exekverbart behövs följande skrivas in i början av scriptet:
\begin{minted}[frame=single,framesep=10pt]{python}
    #!/usr/bin/env python
\end{minted}  
Gör en .py fil är exekverbar  med kommandot:
\begin{minted}[frame=single,framesep=10pt]{bash}
   sudo chmod +x filnamn.py
\end{minted}

Kör i gång programmen för detektion och signalbehandlingen. 
\begin{minted}[frame=single,framesep=10pt]{bash}
    ./detection.py
    ./dsp.py
\end{minted}
Dessa program ligger i /DSP\_final på Odroiden och körs automatiskt i gång när operativsystemet har bootat. Vill man göra tester eller modifikationer måste dessa program avslutas enligt instruktionerna ovan, för att sedan köras i gång efter test eller modifikation.



\begin{itemize}
    \item lista upp saker att göra
    \item länk till github
\end{itemize}

\textbf{Koppla upp systemet}
\begin{enumerate}
    \item Odroiden bootar så fort den får ström.
    \item 2x AirSpy HF+ kopplas till USB 3.0 portarna (blåa).
    \item 2x Koaxialkabel med SMA kontakter kopplas mellan 2x AirSpyHF+ och antennkretsen.
    \item Antennkretsen kopplas i labbmiljö till +12 V DC med ett spänningsaggregat.
\end{enumerate}
\newpage
\textbf{Test av radiomottagaren:}
\begin{enumerate}
    \item Starta Odroid.
    \item Döda de två Python-scripten DSP\_matched.py och detection\_matched.py enligt instruktion ovan.
    \item I terminalen: \textbf{cd DSP\_final} (som är mappen där de två python-scripten ligger)
    \item Editera i detection\_matched.py och ändra \textbf{power\_threshold\_dbm} till något lågt, exempelvis -135. Att editera en fil från terminalen kan göras med \textbf{nano}, exempelvis: \textbf{nano detection\_matched.py}
    \item Se till att 2x AirSpy HF+ är inkopplad i Odroid.
    \item I terminalen: \textbf{./detection\_matched.py} (startar detektions-scriptet) i ett eget terminalfönster.
    \item I terminalen: \textbf{./DSP\_matched.py} (startar GNU Radios signalbehandling) i ett eget terminalfönster.
    \item Nu körs signalbehandling och detektering i varsit terminalfönster, kolla vilka värden i dBm som printas ut i terminalen som kör \textbf{detection\_matched.py} för att få en uppfattning om var brusgolvet befinner sig.
    \item Döda båda scripten igen genom \textbf{CTRL+C} i respektive terminalfönster.
    \item Editera i detection\_matched.py och ändra \textbf{power\_threshold\_dbm} till ett varde som ligger 5 till 10 dBm över brusgolvet som mättes upp i föregående punkt. Exempelvis ändrade jag på Uppsalakontoret värdet till -135 + 45 = -95.
    \item Starta båda scripten igen enligt samma procedur som ovan. 
    \item Nu kan en lavinsändare aktiveras för att detekteras. De detekterade signalstyrkorna printas som tidigare i terminalfönstret där \textbf{detection\_matched.py} körs.
\end{enumerate}
\newpage
Mavlink test:
\begin{itemize}
    \item I mappen DSP\_final ligger Python-scriptet \textbf{mavlink\_test.py}.
    \item För att testa Mavlink-kommunikationen starta scriptet genom att köra följande kommando i terminalen: \textbf{./mavlink\_test.py}.
    \item Nu skickas ett "dummy-meddelande" varje sekund över UART. GPIO pin \#172 (röd sladd). Detta testades med en lokiganalysator och där kunde meddelandena visas. För mer information se mavlinks dokumentation vars länk finns i början av detta dokument.
\end{itemize}

\newpage
\textbf{Depends för gnuradio:}
\inputminted{text}{gnuradio-depends.txt}

\newpage
\textbf{Depends för gr-osmosdr:}
\inputminted{text}{gr-osmosdr-depends.txt}

\newpage
Odroid wiki docs: \url{https://wiki.odroid.com/odroid-xu4/application_note/gpio/uart}

\end{document}